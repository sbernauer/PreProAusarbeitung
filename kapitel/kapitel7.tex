
\addtocontents{toc}{\protect\newpage}
\chapter{Ausblick}
\section{Verbesserungen an PrePro}
An dieser Stelle werden ein paar Punkte genannt, in denen das Projekt PrePro verbessert und weiter entwickelt werden kann.

\subsection{Überladen von Funktionen}
Überladen ist das Definieren von Funktionen mit dem gleichen Namen, die sich nur in der Zahl und Art ihrer Parameter unterscheiden.
So ist es z.B. möglich eine \texttt{add(vec3 a, vec3 b)}-Funktion und eine \texttt{add(mat3 a, mat3 b)}-Funktion zu definieren.
Abhängig davon, ob ein Vektor oder eine Matrix als Parameter übergeben wird, wird die passende Funktion aufgerufen.
Dies ermöglicht es, den Code lesbarer zu gestalten.\\
Aktuell sind die Funktionen in PrePro nicht überladbar, dies könnte aber in einer zukünftigen Arbeit hinzugefügt werden.\\
Für die Umsetzung müsste die \texttt{FunctionTable} angepasst werden.
Diese speichert zur Zeit die Funktionen in einer \texttt{HashMap<String, Function>}.
Diese HashMap muss durch eine eigene Implementierung ersetzt werden, welche die Argumenttypen mit einbezieht.\\
Die Methode\\
\texttt{public Function getFunction(String functionName)} in der \texttt{FunctionTable} müsste um die Argumenttypen als Parameter erweitert werden.
Sie könnte dann z.B. so aussehen:\\
\texttt{public Function getFunction(String functionName, Class... parameterTypes)}

\subsection{Sämtliche Operatoren für die Datentypen implementieren}
Während der Erstellung der Arbeit war es leider zeitlich nicht möglich, alle Operationen, die mit den Datentypen durchgeführt werden können zu implementieren.
Hier muss noch nachgearbeitet werden, damit zukünftigen Nutzern sämtliche Operationen zur Verfügung stehen.\\
Die Operationen sind in den Groovy-Klassen der entsprechenden Datentypen (z.B. \texttt{Vector3}) zu implementieren.
An dieser Stelle sei auf die bisherigen Operatoren als Referenz verwiesen.
Wichtig ist es an dieser Stelle, auch Tests für den Operator zu schreiben, damit seine korrekte Funktionalität dauerhaft sichergestellt werden kann.

\subsection{Performance der Operationen verbessern}
Das Thema Performance ist keine einfaches Gebiet, man kann sich sehr lange damit beschäftigen.
Im Allgemeinen sollte man versuchen, möglichst viele Operationen mit ND4J-Operationen auszudrücken.
Diese wurden von ND4J bereits optimiert und laufen - je nach Umgebung - sogar auf der \ac{GPU}.
Daher sollte es vermieden werden, in einer Operation eine Java-Schleife zu verwenden, da diese zwangsläufig auf der \ac{CPU} ausgeführt wird, und somit den Vorteil der schnelleren \ac{GPU} zunichte macht.\\
Trotzdem gibt in manchen Operatoren Java-Schleifen.
Diese sollten entfernt werden und durch ND4J-Befehle ersetzt werden.

\subsection{Portieren in die Truffle/Graal Architektur}

\subsection{Debugger implementieren}
Wie bereits in \kap{sec:Debugging_graphical} angedeutet wurde, wäre eine Debugger für PrePro-Skripte in der NetBeans-IDE wünschenswert.
Die Gründe und eine Anleitung ist in dem genannten Kapitel verlinkt.

\section{Verbesserungen an der umgebenden Infrastruktur}
Für den Fall, das viele Nutzer PrePro nutzen, kann die umgebende Infrastruktur angenehmer für den Nutzer gestaltet werden.
So senkt sich die Einstiegshürde für neue Benutzer.

\subsection{PrePro-Artefakt ins Maven Central laden}
Alle benötigten Abhängigkeiten werden von Maven standardmäßig aus dem Maven Central geladen.
Es ist eine große Sammlung von Artefakten, welche bequem per Maven-\ac{POM} heruntergeladen und eingebunden werden können.
Ein Hochladen in das Maven Central ermöglicht es, dass Nutzer ganz einfach PrePro in ihren Projekten verwenden können.

\subsection{NetBeans-Plugin in öffentliche Plugin-Sammlung laden}
NetBeans bietet online eine Sammlung von Plugins, welche seht bequem in der NetBeans IDE heruntergeladen und installiert werden können.
Somit entfällt das aufwendige Verteilen einer Datei und das Importieren dieser in NetBeans.

%%%%%%%%%%%%%%%%%%%%%%%%%%%%%%%%%%%%%%%%%%%%%%%%%%%%%%%%%%%%%%%%%%%%%%%%%%%%%%%
\endinput
%%%%%%%%%%%%%%%%%%%%%%%%%%%%%%%%%%%%%%%%%%%%%%%%%%%%%%%%%%%%%%%%%%%%%%%%%%%%%%%
