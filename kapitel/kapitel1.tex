
\chapter{Motivation}
\setcounter{page}{1}
\pagenumbering{arabic}
Immer öfter werden Probleme mittels künstlicher neuronaler Netze gelöst.
Die Daten für das Netz müssen vorher aufbereitet werden.
Dies kann das Normalisierung der Daten und das Entfernen von Rauschen beinhalten.
Diese Maßnahmen können mittels verschiedener Frameworks in verschiedenen Sprachen umgesetzt werden.
Es soll eine \ac{DSL} entwickelt werden, die genau auf diesen Anwendungsfall zugeschnitten ist.
Durch die \ac{DSL} soll eine Sprache geschaffen werden, welche das Vorprozessieren der Daten vereinfacht.
Die Sprache soll plattformübergreifend lauffähig sein und \ac{CPU}- und \ac{GPU}-Berechnungen ermöglichen.

%%%%%%%%%%%%%%%%%%%%%%%%%%%%%%%%%%%%%%%%%%%%%%%%%%%%%%%%%%%%%%%%%%%%%%%%%%%%%%%
\endinput
%%%%%%%%%%%%%%%%%%%%%%%%%%%%%%%%%%%%%%%%%%%%%%%%%%%%%%%%%%%%%%%%%%%%%%%%%%%%%%%
