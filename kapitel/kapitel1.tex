
\chapter{Motivation}
\setcounter{page}{1}
\pagenumbering{arabic}
Immer öfter werden Probleme der realen Welt mittels neuronaler Netze gelöst.
Allerdings kann man in den meisten Fällen nicht einfach das neuronale Netz auf die gemessenen Daten angewendet werden.
Die Daten müssen vorher aufbereitet werden und gegebenenfalls unwichtige Daten entfernt werden.
Dies geschieht mittels verschiedener Frameworks in verschiedenen Sprachen.
Es soll eine \ac{DSL} entwickelt werden, die genau auf diesen Anwendungsfall zugeschnitten ist.
Durch die \ac{DSL} soll einen einheitliche Sprache geschaffen werden, welche das Vorprozessieren der Daten vereinfacht.
Die Sprache soll plattformübergreifend sein und \ac{CPU}- und \ac{GPU}-Berechnungen ermöglichen.

%%%%%%%%%%%%%%%%%%%%%%%%%%%%%%%%%%%%%%%%%%%%%%%%%%%%%%%%%%%%%%%%%%%%%%%%%%%%%%%
\endinput
%%%%%%%%%%%%%%%%%%%%%%%%%%%%%%%%%%%%%%%%%%%%%%%%%%%%%%%%%%%%%%%%%%%%%%%%%%%%%%%
