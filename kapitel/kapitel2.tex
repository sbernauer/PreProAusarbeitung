
\chapter{Überblick über Technologien}

Das Aufbereiten der Daten für das neuronale Netz kann in mehreren Frameworks in mehreren Sprachen erfolgen.
Oft wird für das Vorverarbeiten die gleiche Sprache wie für das neuronale Netz verwendet.
Die gängigsten Frameworks sind in \tab{tab:Frameworks} gelistet.

\begin{table}[H]
	\centering
	\begin{tabular}{ | p{3cm} | p{3cm} | }
		\hline \rowcolor{gray!15}
		\textbf{Framework} & \textbf{Sprache} \\ \hhline{|=|=|}
		Tensorflow & Python \\ \hline
		DL4J & Java \\ \hline
	\end{tabular}
	\caption{Die gängigsten Frameworks für maschinelles Lernen}
	\label{tab:Frameworks}
\end{table}

\section{Unterschied Compiler, Transpiler und Interpreter}
\begin{description}
	\item{\textbf{Compiler}}\newline{Übersetzt von einer höheren Sprache in eine niedrigere Sprache.}
	\item{\textbf{Transpiler}}\newline{Übersetzt zwischen zwei Sprachen mit ungefähr gleichem Abstraktionsgrad.}
	\item{\textbf{Interpreter}}\newline{Führt Code einer höheren Sprache direkt aus.}
\end{description}

\section{ND4J}
ND4J ist eine Framework für die Sprache Java, in welchem effiziente Matrizenoperationen durchgeführt werden können.
Es kann auf der \ac{CPU} oder \ac{GPU} ausgeführt werden.
In ND4J ist größtenteils nur das Konstrukt einer Matrix bekannt, Vektoren sind nur ein Spezialfall einer Matrix.


%%%%%%%%%%%%%%%%%%%%%%%%%%%%%%%%%%%%%%%%%%%%%%%%%%%%%%%%%%%%%%%%%%%%%%%%%%%%%%%
\endinput
%%%%%%%%%%%%%%%%%%%%%%%%%%%%%%%%%%%%%%%%%%%%%%%%%%%%%%%%%%%%%%%%%%%%%%%%%%%%%%%
