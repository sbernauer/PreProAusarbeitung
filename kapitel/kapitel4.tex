
\chapter{Design der DSL}

\section{Funktionen}
In Programmiersprachen werden meist manche Codezeilen häufig benötigt.
Anstatt diese Zeilen mehrfach zu kopieren, kann man diese Zeilen in eine sogenannte Funktion packen.
Diese Funktionen können an beliebiger Stelle im Code aufgerufen werden.
Auf diese Weise kann der entstandene Code gekürzt werden.
Dies erhöht die Lesbarkeit und verhindert Kopierfehler.\\
Daher soll die zu entwickelnde \ac{DSL} auch Funktionen unterstützen.\\
\textbf{Parameter}\\
Funktionen können auch parametrisiert werden, was bedeutet, dass bei dem Aufruf der Funktion Werte mitgegeben werden können.
Jede Funktion hat ihren eigenen Variablen-Gültigkeitsbereich, das bedeutet, dass Funktionen ihren Variablen den gleichen Namen geben können, aber unterschiedliche Variablen verwenden.
Wenn eine Funktion Werte übergeben bekommen möchte, so muss sie diese mitsamt ihrem Typ angeben.\\
\textbf{Rückgabetyp}\\
Funktionen können eine Wert zurückgeben.
Dieser muss einen bestimmten Typ haben.
In der DSL ist ein Rückgabetyp möglich und muss mittels ``returns <Typ>'' gekennzeichnet sein.
Ist keine Angabe gemacht ist keine Rückgabe vorhanden.\\
\textbf{Überladen von Funktionen}\\
Eine Funktion bezeichnet man als überladen, wenn es mehrere Funktionen mit gleichen Namen, aber unterschiedlicher Zahl oder Art von Parametern gibt.
In der DSL ist ein Überladen von Funktionen nicht vorgesehen, könnte aber nachträglich noch implementiert werden.\\
Ein Beispiel von verschiedenen Funktionen befindet sich in \code{lst:Bsp_Funktionen}.\\

\subsection{Main-Funktion in der \acs{DSL}}
Jedes prozedurale Programm benötigt einen Einstiegspunkt, wo das Programm gestartet wird.
Da die DSL ein Framework verwendet, welches in Java geschrieben ist, ist es nicht unwahrscheinlich, dass die zukünftigen Nutzer vorher in Java programmiert haben.
Daher wurde als Einstiegspunkt des Programms - wie in Java - eine Main-Funktion gewählt.
In der DSL besitzt sie keine Parameter.
Um Daten in sein Programm zu laden wurde der Ansatz eines DataSets gewählt, mehr dazu in \kap{sec:Import}.

\subsection{Weglassen der Main-Funktion}
Es gibt mehrere Gründe, warum die Definition einer Main-Funktion unnötig ist:
\begin{itemize}
\item Es gibt Anwendungsfälle in denen ein verkürzte Syntax wünschenswert ist, und auf die Main-Funktion verzichtet werden kann.
\item Kompatibilität mit bisher bestehenden anderen Tools, die keine Funktionen bieten, sondern nur eine Liste von Anweisungen entgegennehmen.
\end{itemize}
Deshalb wurde die Main-Funktion in \acs{PrePro} optional implementiert und muss nicht deklariert werden.
Es reicht aus die Befehle untereinander zu schreiben.\\
Trotzdem wird es als guter Stil erachtet, eine Main-Funktion zu deklarieren.

\begin{lstlisting}[language=prepro, label={lst:Bsp_Funktionen}, caption={Beispiel Funktionen}, captionpos=b]
function main() {
	import vec3 p1, vec3 p2, vec3 p3;

	vec3 x = calculateDifference(p1, p2);
	vec3 s = calculateDifference(p1, p3);
	vec3 y = s X x;
	vec3 z = y X x;


	printResults(x, y, z);

	export x, y, z;
}

function calculateDifference(vec3 p1, vec3 p2) returns vec3 {
	return p2 - p1;
}

function printResults(vec3 x, vec3 y, vec3 z) {
	print x;
	print y;
	print z;
}
\end{lstlisting}

\section{Variablen}
Die \ac{DSL} ist für den Einsatz auf Zeitreihenberechnungen ausgelegt.
Daher stellt in der \ac{DSL} jede Variable eine Zeitreihe dar.
Die Operationen der \ac{DSL} sind immer auf Zeitreihen definiert.\\
Beispielsweise der Ausdruck \texttt{x = a - b;}:
In diesem Fall sind \texttt{a} und \texttt{b} Zeitreihen von Sensordaten.
Die Variable \texttt{x} repräsentiert eine neue Zeitreihe, die durch elementweise Subtraktion jedes Zeitelements entstanden ist.\\
Der Vorteil liegt darin, dass der simple Ausdruck \texttt{x = a - b;} sehr leicht les- und wartbar ist.
Wenn jede Variable keine Zeitreihe, sondern ein einzelner Messpunkt wäre, müsste man eine Schleife verwenden oder sich eigene Methoden definieren bzw. (falls in der Sprache möglich) die Operatoren überschreiben.\\
Variablen haben in der \ac{DSL} immer einen Typ.
Bei dem Anlegen einer Variablen muss dieser auch immer definiert werden.
Ein Typ ist zum Beispiel ein Vector3 (vec3) oder eine Matrix (mat).
Ein Vector3 ist eine Zeitreihe von Vektoren mit der Länge 3, eine Matrix eine Zeitreihe von Matrizen.
Ein dem Programm zur Verfügung gestellter Vektor der Länge 3 kann nun als Vector3 oder auch als Matrix aufgefasst werden.
Daher muss dem Interpreter beim Anlegen der Variablen immer der Typ mitgeteilt werden.
Wenn die Variable schon existiert, muss der Typ nicht erneut angegeben werden.\\
Ein Beispiel befindet sich in \code{lst:Bsp_Variablenzuweisung}.\\
Das Import-Statement wird in \kap{sec:Import} erläutert, relevant ist an dieser Stelle nur, dass mit dem Import Daten aus einem DataSet geladen werden.

\begin{lstlisting}[language=prepro, label={lst:Bsp_Variablenzuweisung}, caption={Beispiel Variablenzuweisung}, captionpos=b]
import vec3 p1, vec3 p2, vec3 p3;

vec3 x = p2 - p1;
vec3 s = p3 - p1;
vec3 y = s X x;
vec3 z = y X x;
\end{lstlisting}

\section{DataSet}
Ein DataSet ist in der \ac{DSL} eine Sammlung von Variablen.
In das DataSet können beliebig viele Zeitreihen und Konstanten unter einem Namen gespeichert werden.
Es ist nicht möglich zwei Elemente mit dem gleichen Namen abzulegen.
Eine Variable kann mittels der Funktion \texttt{INDArray getVariable(String variableName)} aus dem DataSet ausgelesen werden.

\section{Import \& Export}
\label{sec:Import}

Ein Programm, das nur Berechnungen anstellt, ohne ein Ergebnis zu liefern, erscheint auf den ersten Blick unnötig.
Es muss die Möglichkeit haben, Daten zu lesen und zu schreiben.
Im Falle der DSL wird ein eigenes DataSet definiert.
Das Programm erhält bei der Ausführung ein DataSet als Eingabe und gibt als Ausgabe ein DataSet zurück.
In das Eingabe-DataSet werden alle Variablen gespeichert, die für die Berechnungen benötigt werden.
In dem Ausgabe-DataSet sind anschließend alle Variablen gespeichert, die berechnet wurden.\\
Die Verwendung eines DataSet hat gegenüber dem Übergeben mittels Parametern in die Main-Funktion folgende Vorteile:
\begin{itemize}
\item Es gibt nur einen Rückgabetyp (DataSet).
Andernfalls müsste ein Konstrukt ersonnen werden, mehrere Objekte (z.B. Zeitreihen oder Konstanten) von der Main-Funktion zurückgeben zu lassen.
\item Einfacher Aufruf der Main-Funktionen (ab 4 Parametern wird der Funktionsaufruf unübersichtlich\cite{parameterCount}). Statt 20 Parameter zu übergeben kann übersichtlich das DataSet zusammengebaut werden und als einziges Argument übergeben werden.
\item Einfaches ``Weiterreichen'' von DataSets zwischen mehreren PrePro-Programmen.
Falls mehrere PrePro-Programme nacheinander ausgeführt werden, kann das Ausgabe-DataSet des ersten Programms als Eingabe-DataSet des zweiten Programms genommen werden.
\end{itemize}

\subsection{Optionale Imports}
Es soll möglich sein, Importe in die Main-Funktion als optional zu kennzeichnen.
Diese werden mit dem Schlüsselwort \texttt{optional} gekennzeichnet.
Eine Beispiel mit optionalen Imports ist in \code{lst:OptionalImport} gegeben.
Wenn die angegebene, zu importierende Variable nicht in dem DataSet existiert, wird keine Fehlermeldung erzeugt, sondern die Variable nicht importiert.\\
\textbf{Achtung}: Wenn ein Import optional ist und die Variable nicht in dem DataSet existierte, wurde die Variable nicht geladen.
Wenn nun in dem Programm versucht wird auf die Variable zuzugreifen wird eine Fehlermeldung erzeugt, dass die Variable nicht definiert sei.
Deshalb ist es wichtig, die Gültigkeit der Variablen mittels der \texttt{exists}-Funktion zu prüfen, die im nächsten Kapitel vorgestellt wird.
Wenn bei der Konstruktion des Ausführungspfad nicht aufgepasst wird, kann auf eine nicht existierende Variable zugegriffen werden.
Das ist vergleichbar mit Nullpointern, die in eine Funktion hereingegeben werden können und zunächst mal auf \texttt{!= null} geprüft werden müssen.\\
Deshalb ist es definitiv nicht ratsam, alle Imports als optional zu kennzeichnen.
Vielmehr sollten nur die wirklich optionalen Variablen als optional gekennzeichnet werden.

\begin{lstlisting}[language=prepro, label={lst:OptionalImport}, caption={PrePro-Code mit optionalen Imports}, captionpos=b]
function main() {
	import vec3 p1, optional const quaternion_1, optional mat4 notPresent;

	const isPresent = exists(p1);
	const isNotPresent = foo();

	export p1, quaternion_1, isPresent, isNotPresent;
}

function foo() returns const {
	return exists(notPresent);
}
\end{lstlisting}

\subsection{Die exists-Funktion}
Mittels der Funktion \texttt{exists(<VariablenName>)} kann abgefragt werden, ob diese Variable geladen wurde.
Wurde die Variable zur Verfügung gestellt, gibt die Funktion eine 1 zurück, andernfalls eine 0.
Nach Abschluss der Berechnungen in \code{lst:OptionalImport} hat die Variable \texttt{isPresent} der Wert 1, die Variable \texttt{not Present} den Wert 0.
Dies ist in folgendem Fall nützlich:
In ein Programm werden zwei Koordinaten-Zeitreihen A und B gegeben.
Im Regelfall wird auf dem Mittelpunkt zwischen diesen Punkten gearbeitet.
Ist jedoch die Variable A nicht definiert, soll für die Berechnung B verwendet werden, und anders herum.
Eine kleine Abweichung wird hier in Kauf genommen.
Dieser Fall mag ungewöhnlich wirken, warum sollte der Code nicht einfach an die Verwendung von A oder B angepasst werden?
Dies gibt mehr Sinn, wenn ein Skript allgemein verwenden werden soll um häufig genutzte Berechnungen auszulagern.
In diesem Fall möchte man nicht jedes Mal die Berechnung aufs neue anpassen, sondern die Funktion einmal allgemein schreiben und die Ausführung abhängig von den gegebenen Daten machen.
Das oben genannte Beispiel mit A und B könnte in etwa so ausgedrückt werden.
Der Code in \code{lst:OptionalImportExample}, soll das Prinzip verdeutlichen.
\begin{lstlisting}[language=prepro, label={lst:OptionalImportExample}, caption={PrePro-Code mit optionalen Imports}, captionpos=b]
import optional vec3 a, optional vec3 b;

vec3 center = (exists(a) && exists(b)) ** (a + b) / 2
|| exists(a) ** a
|| exists(b) ** b
|| throw "Neither a or b exists";

export center;
\end{lstlisting}
Falls die Variable a und b beide existieren ergibt der Ausdruck \texttt{(exist(a) \&\& exist(b))} eine 1, andernfalls eine 0.
Wichtig ist an dieser Stelle, dass die Multiplikationen lazy sind:
Falls a existiert und b nicht, ergibt der Ausdruck \texttt{(exist(a) \&\& exist(b))} eine 0, der Interpreter rechnet dann weiter, also \texttt{* ( a + b ) / 2}.
Das Ergebnis einer Multiplikation ist immer null, soweit ist die Rechnung korrekt, da durch das \texttt{||} die anderen Fälle probiert werden.
Das Problem ist:
Der Interpreter versucht \texttt{* ( a + b ) / 2}. zu berechnen und erzeugt - vollkommen zurecht - eine Fehlermeldung, dass b nicht definiert ist.\\
Zur Lösung dieses Problems gab es 2 Möglichkeiten:
\begin{enumerate}
\item \textbf{Lazy Multiplikation}\\
Es wird ein neuer Multiplikations-Operator eingeführt, der den rechten Teil nur berechnet, wenn der linke Teil != 0 ist.
Dadurch würde nach dem Teil \texttt{(exist(a) \&\& exist(b))} die Berechnung abgebrochen, da der linke Teil 0 ergibt.
Durch das Abbrechen wird nicht versucht Variablen zu verwenden, die nicht definiert sind.
Für diesen neuen Multiplikations-Operator wurde ein neues Symbol definiert: ``**''.\\
\textbf{Vorteile}:
\begin{itemize}
\item Einfache Implementierung.
\item Wiederverwendung des Operators an anderer Stelle möglich.
\end{itemize}
\textbf{Nachteile}:
\begin{itemize}
\item Die Nutzer müssen einen zusätzlichen Operator kennen und verstehen.
\end{itemize}

\item \textbf{Dedizierte If-Statements}\\
In die \ac{DSL} PrePro wird noch das Konstrukt einer If-Verzweigung eingebaut.
Der beispielhafte PrePro-Code würde somit wie in \code{lst:OptionalImportExample_If} aussehen.
\begin{lstlisting}[language=prepro, label={lst:OptionalImportExample_If}, caption={PrePro-Code mit If-Statements und der exists-Funktion}, captionpos=b]
import optional vec3 a, optional vec3 b;

if(exists(a) && exists(b)) {
    vec3 center = ( a + b ) / 2;
} else if (exists(a)) {
    vec3 center = a;
} else if (exists(b)) {
    vec3 center = b;
}
<<OPTIONAL>>
else {
	throw "Neither a or b exists";
}
\end{lstlisting}
\textbf{Vorteile}:
\begin{itemize}
\item Leichter lesbar und verständlich.
\item Auch für andere Fälle als die Multiplikation ist das If-Statement verwendbar.
\end{itemize}
\textbf{Nachteile}:
\begin{itemize}
\item Erhöht die Anzahl der Codezeilen dramatisch.
\item Höherer Aufwand bei der Implementierung
\end{itemize}
\end{enumerate}
Aufgrund der einfacheren Implementierung und der kurzen, prägnanten Ausdrucksweise wurde die Variante der Lazy Multiplikation in PrePro umgesetzt.

\section{Kommentare}
In der \ac{DSL} wird einem Zeilen-Kommentar ein ``//'' vorangestellt.\\
Ein Block-Kommentar wird mit ``/*'' und ``*/'' umschlossen.

%%%%%%%%%%%%%%%%%%%%%%%%%%%%%%%%%%%%%%%%%%%%%%%%%%%%%%%%%%%%%%%%%%%%%%%%%%%%%%%
\endinput
%%%%%%%%%%%%%%%%%%%%%%%%%%%%%%%%%%%%%%%%%%%%%%%%%%%%%%%%%%%%%%%%%%%%%%%%%%%%%%%
