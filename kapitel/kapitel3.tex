\chapter{Aufgabenstellung}
\section{Zu Grunde liegendes Framework}
In dem Umfeld der Arbeit hat sich das Framework ND4J in der Programmiersprache Java für den Praxiseinsatz durchgesetzt.
Daher soll die in dieser Arbeit entwickelte \ac{DSL} auf diesem Framework aufbauen.

\section{Compiler und Interpreter}
Die \ac{DSL} ist für das Vorprozessieren von Daten für maschinelles Lernen.
Für die \ac{DSL} kommt ein Compiler oder Interpreter in Frage.
Ein Transpiler ist nicht möglich, da es keine in der Praxis verwendete Sprache für das Vorprozessieren der Daten gibt, in die übersetzt werden kann.
In der \tab{tab:Vorteile_Compiler_Interpreter} werden die Implementierungsmöglichkeiten mittels Compiler und Interpreter gegenüber gestellt.\\
Wegen der Vorteile (vornehmlich das Debugging) von Interpretern im Vergleich zu Compiler soll PrePro als Interpreter implementiert werden.

\begin{table}[H]
	\centering
	\begin{tabular}{ | p{3cm} | p{6cm} | p{6cm} | }
		\hline \rowcolor{gray!15}
		\textbf{Implemen-tierung} & \textbf{Vorteile} & \textbf{Nachteile} \\ \hhline{|=|=|=|}
		Compiler & Generierter Java-Code kann auf jeder \ac{JVM} ausgeführt werden, es wird kein Interpreter benötigt. & Debugging ist nur in dem generierten Java-Code möglich. \\ \hline
		Interpreter & Debugging leichter möglich & Möglicherweise nicht so performant \\ \hline
	\end{tabular}
	\caption{Vor- und Nachteile einer Implementierung mittels Compiler oder Interpreter}
	\label{tab:Vorteile_Compiler_Interpreter}
\end{table}

\label{sec:usedTechnologies}
\section{Verwendete Technologien}
Die \ac{DSL} wird mittels einem Interpreter ausgeführt.
Dieser ist in Java (genauer: Groovy) geschrieben und baut auf folgenden Technologien auf:
\begin{description}
	\item{\textbf{ND4J}}\newline Matrizen-Berechnungen werden mittels dem ND4J-Framework durchgeführt.
	\item{\textbf{Java}}\newline Das ND4J-Framework ist in der Programmiersprache Java verfügbar. Damit der Interpreter es verwenden kann, wird in dieser Sprache geschrieben.
	\item{\textbf{Groovy}}\newline Groovy ist eine Sprache, die auf Java aufbaut und kompatibel ist. Sie unterstützt zum Beispiel dynamic dispatching\footnote{\url{https://en.wikipedia.org/wiki/Dynamic_dispatch}}.
	\item{\textbf{Antlr}}\newline Antlr in der Version 4 wird für das Parser der eingegebenen Programme verwendet. Für das Netbeans-Plugin wird Antlr in der Version 3 verwendet.
\end{description}

%%%%%%%%%%%%%%%%%%%%%%%%%%%%%%%%%%%%%%%%%%%%%%%%%%%%%%%%%%%%%%%%%%%%%%%%%%%%%%%
\endinput
%%%%%%%%%%%%%%%%%%%%%%%%%%%%%%%%%%%%%%%%%%%%%%%%%%%%%%%%%%%%%%%%%%%%%%%%%%%%%%%
